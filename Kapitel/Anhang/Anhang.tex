
\chapter*{Anhang}
\addcontentsline{toc}{chapter}{Anhang}


\section*{Pflichtenheft}
\addcontentsline{toc}{section}{Pflichtenheft}

Bei klassischen vertraglichen Einigungen definieren Auftragnehmer und Auftraggeber in der Regel ein gemeinsames Lastenheft. Aus diesem wird dann ein Pflichtenheft erstellt, welches die Lasten in technische Umsetzungen wandelt. Vor Vertragsbeginn muss der Auftraggeber dieses nun abnehmen. Da dies beim Prototypen nicht umgesetzt wurde, da eine Einigungen zu den zu erledigenden Pflichten innerhalb einer Abschlussarbeit nur bedingt sinnvoll ist, wird im folgenden das Lastenheft erneut aufgenommen und auf seine Umsetzung untersucht. Dies soll als Pflichtenheft dienen.

\begin{table}[H]
\begin{center}
  \begin{tabular}{| l | l | l | l |}

\hline
\rowcolor{Gray}
\textcolor{white}{\textbf{Ref.}} & \textcolor{white}{\textbf{Anforderung}} & \textcolor{white}{\textbf{Quelle}} & \textcolor{white}{\textbf{Erledigt}} \\

\hline
\rowcolor{LGray}
1		& Plattform						& Auftraggeber &\\
\hline
1.1		& Betriebssystemunabhängigkeit 	& Auftraggeber & Ja\\
\hline
1.2	 	& Geräteunabhängigkeit			& Auftraggeber & Ja\\
\hline
1.3		& Mobile First					& Auftraggeber & Teils\\
\hline

\hline    
\rowcolor{LGray} 						
2		& Stundenplan 										& Auftraggeber &\\
\hline
2.1		& Abrufen des allgemeinen 							& Auftraggeber & Ja\\
		& Stundenplans										&			   &\\
\hline
2.2		& Personalisierung nach Fakultät					& Auftraggeber & Ja\\
\hline
2.3		& Personalisierung nach Studiengang					& Auftraggeber & Ja\\
\hline
2.4		& Personalisierung nach Fachsemester 				& Auftraggeber & Ja\\
\hline
2.5		& Einbindung nicht regulärer 						& Auftraggeber & Nein\\
		& Vorlesungen										&			   &\\
\hline
2.6		& Einbindung des erweiterten 						& Auftraggeber & Teils\\
		& Vorlesungsspektrums								&			   &\\
\hline

\end{tabular}
  \end{center}
\caption[Pflichtenheft]{Pflichtenheft}
\label{tab:lastenheft}
\end{table}

\begin{table}[H]
\begin{center}
  \begin{tabular}{| l | l | l | l |}
  
\hline
\rowcolor{Gray}
\textcolor{white}{\textbf{Ref.}} & \textcolor{white}{\textbf{Anforderung}} & \textcolor{white}{\textbf{Quelle}} & \textcolor{white}{\textbf{Erledigt}} \\

\hline    
\rowcolor{LGray} 						
3		& Stundenplan Änderungen								& Auftraggeber & \\
\hline
3.1		& Unterscheidung einmalige/ 							& Auftraggeber & Nein\\
		& langfristige Änderungen								&			   &\\
\hline
3.2		& Automatische Anpassung langfristiger 			 		& Auftraggeber & Ja\\
		& Änderung in Stundenplan								& 			   &\\
\hline
3.3		& Benachrichtigung der Nutzer 							& Auftraggeber & Teils\\
		& bei Änderungen										& 			   &\\
\hline
3.4		& Anzeige einmaliger Änderungen 						& Auftraggeber & Ja\\
		& in Stundenplan										&			   &\\
\hline

\hline    
\rowcolor{LGray} 						
4		& Speiseplan											& Auftraggeber & \\
\hline
4.1		& Anzeige des Speiseplans 								& Auftraggeber & Ja\\
		& (Studentenwerk Oberfranken)							&			   &\\
\hline
4.2		& Filterung der Anzeige von 							& Auftraggeber & Ja\\
		& Speiseplan Informationen								&			   &\\			
\hline
4.3		& Anzeige zusätzlicher Informationen 			 		& Auftraggeber & Ja\\
		& pro Gericht (z.B. Allergene)							& 			   &\\
\hline

\hline    
\rowcolor{LGray} 						
5		& Anwenderverwaltung								& Auftraggeber &\\
\hline
5.1		& Speicherung Nutzer spezifischer 					& Auftraggeber & Ja\\
		& Informationen										&			   &\\
\hline
5.2		& Anmeldung notwendig bei 						 	& Auftraggeber & Ja\\
		& Speicherung der Daten								&			   &\\
\hline
5.3		& Aufteilung der Nutzer in 							& Auftraggeber &Teils\\
		& verschiedene Gruppen								&			   &\\
\hline
5.4		& Anmeldung durch Hochschul-E-Mail-					& Auftraggeber & Ja\\
		& Adresse											&			   &\\
\hline
5.5		& Anmeldung auch für Studierende 					& Auftraggeber & Teils\\
		& ohne FH-E-Mail									& 			   &\\
\hline
5.6		& Automatisiertes Löschen alter Daten				& Auftraggeber & Teils\\
\hline

  \end{tabular}
  \end{center}
\caption[Pflichtenheft]{Pflichtenheft}
\label{tab:lastenheft}
\end{table}

\begin{table}[H]
\begin{center}
  \begin{tabular}{| l | l | l | l |}
  
\hline
\rowcolor{Gray}
\textcolor{white}{\textbf{Ref.}} & \textcolor{white}{\textbf{Anforderungen}} & \textcolor{white}{\textbf{Quelle}} & \textcolor{white}{\textbf{Erledigt}} \\

\hline    
\rowcolor{LGray} 						
6		& Mehrsprachigkeit									& Auftraggeber/		& \\ \rowcolor{LGray}
		&													& International-	&\\ \rowcolor{LGray}
		&													& Office			&\\ 
\hline
6.1		& App in deutscher Sprache							& Auftraggeber 		& Ja\\
\hline
6.2		& App in englischer Sprache 						& Auftraggeber/		& Ja\\
		&													& International-	&\\
		&													& Office 			&\\
\hline
6.3		& Einfache Erweiterung der App 						& International-	& Ja\\
		& um weitere Sprachen								& Office 			&\\
\hline

\hline
\rowcolor{LGray} 						
7		& Einfache personalisierte 							& Auftraggeber/		&\\ \rowcolor{LGray}
		& Stundenplanerstellung  							& International-	&\\ \rowcolor{LGray}
		&													& Office			&\\
\hline
7.1		& Fakultäten unabhängige 							& Auftraggeber/		& Ja\\
		& Stundenplanerstellung								& International-	&\\
		&													& Office 			&\\ 
\hline
7.2		& Studiengang unabhängige					 		& Auftraggeber/		& Ja\\
		& Stundenplanerstellung								& International-	&\\
		&													& Office			&\\
\hline
7.3		& Einfache Einbindung externer Kurse				& Auftraggeber/		& Teils\\
		&													& Sprachzentrum	    &\\
\hline    

\hline
\rowcolor{LGray} 						
8		& Einbindung des Sprachenzentrums 						& Sprachzentrum &\\
\hline
8.1		& Einbinden von Sprachkursinformationen					& Sprachzentrum &Ja\\
\hline
8.2		& Mehrsprachige Sprachkursinformationen					& Sprachzentrum &Teils\\
\hline
8.3		& Einheitliche Darstellung der 							& Sprachzentrum &Ja\\
		& Sprachkursinformationen								& 				&\\
\hline
8.4		& Vollständige Informationsdarstellung					& Sprachzentrum & Nein\\
\hline

  \end{tabular}
  \end{center}
\caption[Pflichtenheft]{Pflichtenheft}
\label{tab:lastenheft}
\end{table}

\newpage

\section*{Autoren Referenz}
\addcontentsline{toc}{section}{Autoren Referenz}

Diese wissenschaftliche Arbeit ist eine Zusammenarbeit der beiden Autoren Dennis Brysiuk und Noah Lehmann. Um daher einen besseren Überblick über die Zuordnung der Inhalte zu haben wird im folgenden eine Tabelle dargestellt, die die Kapitel ihren jeweiligen Verfassern zuordnet.

\begin{table}[H]
\begin{center}
  \begin{tabular}{| l | l | l |}
 
\hline
\rowcolor{Gray}
\textcolor{white}{\textbf{Kapitel}} & \textcolor{white}{\textbf{Kapitel Bezeichnung}} & \textcolor{white}{\textbf{Autor}} \\
\rowcolor{Gray}
\textcolor{white}{\textbf{Nr.}} 	&  												  & \\  

\hline    
\rowcolor{LGray} 						
\textbf{1}		& \textbf{Einleitung}	&  				\\
\hline
1.1		& Beweggründe					& Noah Lehmann	\\
\hline
1.2		& Zielsetzung					& Noah Lehmann	\\
\hline
1.3		& Zielgruppe					& Noah Lehmann	\\
\hline
1.5		& Vorgeschlagene Nebenlektüre	& Noah Lehmann	\\

\hline    
\rowcolor{LGray} 						
\textbf{2}		& \textbf{Vorgehensweise} &  				\\
\hline
2.1		& Grundkonzept								& Dennis Brysiuk	\\
\hline
2.2		& Umsetzung									& Noah Lehmann	\\

\hline    
\rowcolor{LGray} 						
\textbf{3}	& \textbf{Technologien} &  				\\
\hline
3.1		& Programmiersprache					& Noah Lehmann	\\
\hline
3.2		& Microservice Framework				& Dennis Brysiuk\\
\hline
3.3		& Build Management Tool					& Noah Lehmann	\\
\hline
3.4		& Versionsverwaltung Tool				& Noah Lehmann	\\
\hline
3.5		& API-Gateway							& Dennis Brysiuk\\
\hline
3.6		& Service Registrierung und Discovery	& Dennis Brysiuk\\
\hline
3.7		& Benachrichtigung						& Dennis Brysiuk\\

\rowcolor{LGray} 						
\textbf{4}	& \textbf{Konfiguration} &  				\\

\hline
4.1		& Entwicklungsumgebung		& Noah Lehmann	\\
\hline
4.2		& Spring Boot				& Dennis Brysiuk	\\
\hline
4.3		& Spring Cloud Gateway		& Dennis Brysiuk	\\
\hline
4.4		& Eureka					& Dennis Brysiuk	\\
\hline
4.5		& API Filter				& Noah Lehmann	\\
\hline
4.6		& Error Handling 			& Noah Lehmann	\\
\hline
4.7		& Firebase Cloud Messaging	& Dennis Brysiuk	\\


\hline
  \end{tabular}
  \end{center}
\caption[Autoren Referenz]{Autoren Referenz}
\label{tab:autoren}
\end{table}

\newpage

\begin{table}[H]
\begin{center}
  \begin{tabular}{| l | l | l |}
 
\hline
\rowcolor{Gray}
\textcolor{white}{\textbf{Kapitel}} & \textcolor{white}{\textbf{Kapitel Bezeichnung}} & \textcolor{white}{\textbf{Autor}} \\
\rowcolor{Gray}
\textcolor{white}{\textbf{Nr.}} 	&  												  & \\  

\hline    
\rowcolor{LGray} 						
\textbf{5}	& \textbf{Entwicklungsumgebungen}	&	\\
\hline
5.1		& Dev-Environment					& Dennis Brysiuk \\
\hline
5.2		& Prod-Environment					& Noah Lehmann \\

\hline    
\rowcolor{LGray}
\textbf{6}		& \textbf{Microservice Dokumentation}	&  			\\
\hline
6.1		& Konfiguration im Microservice			& Noah Lehmann \\
\hline
6.2		& Controller Beschreibung				& Noah Lehmann \\
\hline
6.3		& Dokumentation der Funktionen			& Noah Lehmann \\
\hline
6.4		& Export der Dokumentation				& Dennis Brysiuk \\
\hline
6.5		& Nutzen der grafische Oberfläche		& Dennis Brysiuk \\

\hline    
\rowcolor{LGray}
\textbf{7}		& \textbf{Testing}	&  				\\
\hline
7.1		& Controller Tests				& Dennis Brysiuk \\
\hline
7.2		& Service Tests					& Dennis Brysiuk \\
\hline
7.3		& Persistenz Tests				& Dennis Brysiuk \\
\hline
7.4		& Funktionstests				& Noah Lehmann \\

\hline    
\rowcolor{LGray} 						
\textbf{8} & \textbf{Probleme} &  				\\
\hline
8.1		& Kontinuierliche Integration und Deployment	& Noah Lehmann \\
\hline
8.2		& HATEOAS										& Dennis Brysiuk \\
\hline
8.3		& Notification Service							& Dennis Brysiuk \\

\hline    
\rowcolor{LGray} 						
\textbf{9} & \textbf{Weiterentwicklung}	&  				\\
\hline
9.1		& Pflege der Anwendung					& Noah Lehmann \\
\hline
9.2		& Programmcode Veröffentlichung			& Noah Lehmann \\
\hline
9.3		& Verteilung von Zugangsdaten			& Dennis Brysiuk \\
\hline
9.4		& Problemfindung und Beseitigung		& Dennis Brysiuk \\
\hline
9.5		& Neue Funktionen						& Dennis Brysiuk \\

\hline    
\rowcolor{LGray} 						
\textbf{10} & \textbf{Ausblick und Fazit} &  				\\
\hline
10.1		& Ausblick						& Noah Lehmann \\
\hline
10.3		& Fazit							& Noah Lehmann \\
\hline


\hline
  \end{tabular}
  \end{center}
\caption[Autoren Referenz]{Autoren Referenz}
\label{tab:autoren}
\end{table}


\cleardoublepage