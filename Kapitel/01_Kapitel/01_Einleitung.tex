% Einleitung
\chapter{Einleitung}
\label{sec:einleitung}
% Text
Um ein Projekt sinnvoll durchführen zu können und um das nach der Vollendung entstandene Ergebnis auch weiter pflegen und entwickeln zu können, bedarf es einer strukturierten Dokumentation zur Vorgehensweise im Verlauf des Projekts. Hierbei müssen jedoch einige Dinge beachtet werden: sind die Projektplanungen bereits abgeschlossen, steht schon die Grundstruktur des Projekts und was genau muss eigentlich dokumentiert werden?\\
\linebreak
All das soll in diesem Kapitel kurz erläutert werden. Das Projekt, um das es sich handelt, ist der Prototyp der neuen web-basierten Hochschul-\ac{App}. Diese wird im Verlaufe dieser Arbeit parallel entwickelt, weshalb die Arbeit diesem Prototypen als Dokumentation und Rechtfertigung der genutzten Techniken dienen soll. Die folgenden Kapitel sollen nun darauf eingehen, weshalb solch eine Dokumentation von Nöten ist, was in dieser Dokumentation alles abgehandelt werden soll, welche Leser-Gruppen schlussendlich von ihr profitieren können und welche Grundkenntnisse und Nebenlektüre zum Lesen dieser Arbeit sinnvoll sind.\\
\linebreak
Im Anschluss soll kurz auf die Vorgehensweise bei der Erstellung des Prototypen eingegangenen werden. Hierbei ist keine genaue Projektplanung von Nöten, stattdessen sollen nur kurz die Konzepte vorgestellt werden, nach denen gearbeitet wird. Sollten gewisse Techniken genannt werden, so werden diese auch später in der Arbeit genauer erläutert. Auch die hier nicht genannten Technologien sollen aber noch genauer erläutert werden, denn nur so kann ein volles Verständnis dafür entwickelt werden, weshalb gewissen Technologien ihren Alternativen vorgezogen wurden.\\
\linebreak
Nachdem ein allgemeines Verständnis der Technologien vermittelt wurde, soll nun genauer auf die Implementierung der neuen Hochschul-\ac{App} eingegangen werden. Hierbei werden nun technische Details bereitgestellt, die vor allem neuen Entwicklern einen leichten Einstieg in die Implementierung bieten sollen. Um diesen Entwicklern noch weitere Werkzeuge an die Hand zu geben wird kurz auf die empfohlene Entwicklungsumgebung eingegangen. Hierbei soll vor allem beleuchtet werden, welche nützlichen Tools eingerichtet wurden, um die Entwicklungsarbeit an dem Prototypen stark zu vereinfachen.\\
\linebreak
Anschließend kann nun die vorhandene Implementierung genauer betrachtet werden. Hierbei soll weniger auf den detaillierten Programmcode, sondern eher auf die Bereitstellung der Ressourcen des \ac{REST}-Services geachtet werden. Es werden alle Microservices und ihre Feinheiten betrachtet, sodass am Ende ein klares Verständnis über den Aufbau der Datenschnittstelle vermittelt wurde. Neben der Dokumentation der Services fallen natürlich auch die Tests zu genau dieser an. Diese sind besonders wichtig und dringend zu dokumentieren, denn an den bereitgestellten Tests können sich spätere Entwickler richten, wenn sie neue Funktionen implementieren. So kann klargestellt werden, dass die Qualität der neuen Hochschul-\ac{App} konstant bleibt, auch wenn sich das Entwicklerteam über längere Zeit verändert.\\
\linebreak
Zum Abschluss sollen noch die Probleme bei der Implementierung genauer betrachtet werden. Dabei soll es um nicht umgesetzte oder nicht umsetzbare Funktionen gehen. Das soll vor allem den späteren Verlauf der Entwicklung unterstützen, denn hier werden auch allgemeine Probleme festgehalten. Nachdem die Probleme festgehalten wurden soll noch die spätere Weiterentwicklung geplant werden. Hierbei werden Werkzeuge und Konzepte bereitgestellt, die eine Zusammenarbeit von vielen Entwicklern und auch interessierten Dritten ermöglicht. Abschließend daran soll das Gesamtprojekt noch einmal reflektiert werden.

\section{Beweggründe}

Bei dem initialen Projekt zur Neuaufsetzung der Hochschul-\ac{App} als web-basierte Anwendung ist viel Aufwand in die Analyse der Möglichkeiten und der Zusammenstellung der genutzten Techniken geflossen. Es wurden Umfragen gestartet, Anforderungen aus verschiedenen Quellen gesammelt und anschließend, aufbauend auf den gewonnenen Erkenntnissen, eine aufwändige Architektur gestaltet. Um diese Architektur schließlich sinnvoll umsetzen zu können wurden einige Techniken und Frameworks analysiert, welche einer Erklärung bedürfen. Die genannten Schritte bei der Analysearbeit wurden in der zu dieser Arbeit parallel erstellten Bachelorarbeit dokumentiert\autocite[][]{dnba}.\\
\linebreak
Jedoch reicht bei der Erstellung eines Prototypen die reine Analyse und die Sammlung der Anforderungen nicht aus. Am Ende muss auch ein lauffähiges Programm entstehen, welches dann ausgiebig getestet und anschließend weiterentwickelt werden kann. Auch aus den im Entwicklungsprozess einer Anwendung gewonnenen Erkenntnissen lassen sich im Nachhinein hilfreiche Schlüsse ziehen, weshalb bei der Durchführung eines Projektes immer eine Praxisarbeit zum dokumentieren des Vorgehens angefertigt werden sollte. Genau dazu dient diese Arbeit. Sie soll den Entwicklungsprozess des Projektes der \textit{web-basierten Hochschul-\ac{App}} aufzeichnen und die verwendeten Techniken, Prinzipien, Frameworks aber auch Vorgehensweisen festhalten, damit spätere Entwicklerteams einen leichten Einblick in die Ideen und Lösungen der Entwickler des Prototypen der neuen Hochschul-\ac{App} erhalten können.

\section{Zielsetzung}

Wie bereits erwähnt soll diese Arbeit als begleitendes Handbuch zur Implementierung des Prototypen der \textit{web-basierten Hochschul-\ac{App}} dienen. Das bedeutet nicht nur, dass hier die genutzten Techniken genauer erläutert werden, sondern auch die Herangehensweise an das Projekt, die Schwierigkeiten, die die Implementierung mit sich gebracht hat und die Vorkehrungen, die für spätere Entwicklerteams getroffen wurden.\\
\linebreak
Im Laufe dieser Arbeit soll die Vorgehensweise, mit der die Anwendung umgesetzt wurde, klar dargestellt werden. Das beinhaltet nicht nur die reine Vorgehensweise, mit der gearbeitet wurde, sondern auch die Organisation zwischen den Entwicklern untereinander. Des weiteren sollen die genutzten Technologien genau erläutert werden. Hier soll vor allem auch darauf eingegangen werden, weshalb die Technologien ihren jeweiligen Alternativen vorgezogen wurden und wie die Technologien genau in den Prototypen der Hochschul-\ac{App} eingeflossen sind. Sobald die Technologien dann wiederum ausreichend erklärt wurden soll die eigentliche Implementierung der web-basierten Hochschul-\ac{App} genauer betrachtet werden. Mit dem gewonnenen Grundwissen aus den erklärten Technologien soll dem Leser nun die Konfiguration der einzelnen Komponenten deutlich gemacht werden. Dieser Teil der Arbeit ist eher Quellcode-lastig und deshalb vor allem für spätere Entwicklerteams interessant.\\
\linebreak
Ebenso für die Entwicklerteams sinnvoll zu lesen sind die darauf folgenden Kapitel. Sie handeln von den Umgebungen, die den Entwicklern beim Programmieren und Testen zur Verfügung stehen. Hier anzumerken ist, dass bei der Entwicklung des Prototypen viel Wert darauf gelegt wurde, dass später schnell neue Entwickler in das Projekt eingegliedert werden können. Auch die exakte Dokumentation der Endpunkte der einzelnen Services der Hochschul-\ac{App}-Datenbereitstellung ist von besonderem Interesse für Entwickler. Denn nach dieser Dokumentation können sich vor allem auch Frontend Programmierer richten, die lediglich die Datenbereitstellung nutzen, um eine eigene Anwendung zu erstellen. Anschließend an die eigentliche Dokumentation wird kurz auf das Testverfahren für das Backend der Hochschul-\ac{App} eingegangen.\\
\linebreak
Abschließend sollen die Leser dieser Arbeit noch erkennen, welche Vorkehrungen getroffen wurden, um die \ac{App} nach Abschluss des Prototypen weiterentwickeln zu können. Dabei werden einige Punkte angesprochen, die die Einbindung von neuen Entwicklern unterstützen und vor allem auch die Verbesserung der Anwendung im Allgemeinen vorantreiben sollen. Dieser Teil soll dann auch als Abschluss der Arbeit dienen und in einem kurzen Fazit zur Durchführung des Projektes enden.

\section{Zielgruppe}

Wie bereits mehrfach erwähnt wurde richtet sich diese Arbeit zu großen Teilen den Entwicklern, die später weiter an dem Projekt der neuen Hochschul-\ac{App} arbeiten werden. Diese werden in dieser Dokumentation zum Prototypen einige Erklärungen zu verwendeten Techniken, Anleitungen zur Verwendung gewisser Technologien und Richtlinien zum Arbeiten am Programmcode finden. Wie bereits zu vermuten ist, sollten diese Entwickler bereits ein fundiertes Grundwissen in den Bereichen der Webtechnologien und in der Software Entwicklung im Allgemeinen vorweisen können.\\
\linebreak
Jedoch ist diese Praxisarbeit nicht nur an die Entwickler der neuen Hochschul-\ac{App} gerichtet, sondern auch an alle Beteiligten, die ein Interesse am Aufbau und den benötigten Ressourcen dieser Anwendung haben. So wird der Einsatz gewisser Technologien gerechtfertigt, was spätere Entscheidungen bei Änderungen der Lizenzbedingungen dieser Technologien erleichtern soll. Des weiteren werden für solche Fälle stets Alternativen genannt. Die Teile der Arbeit, die für außenstehende, nicht-Entwickler geeignet sind, sind auch stets in abgeschwächter Fachsprache verfasst, sodass jeder sie verstehen kann. Für diese Teile soll kein fundiertes Grundwissen aus der Informatik von Nöten sein.

\section{Vorgeschlagene Nebenlektüre\label{sec:nebenlektuere}}

Da es sich bei dieser Arbeit um eine Praxisarbeit handelt, die parallel zu Analyse und Implementierung des Prototypen der neuen, web-basierten Hochschul-\ac{App} angefertigt wurde, bietet es sich zu aller erst an, die weiteren Arbeiten zu lesen, die im Zuge dieses Projektes erstellt wurden. Allen voran ist hierbei die Analyse zur Anforderungssammlung, der Architektur und der \ac{REST}-Schnittstelle zu erwähnen, welche in der Bachelorarbeit \textit{Web-basierte Hochschul-\ac{App} - modulare Web-Architektur} ausführlich behandelt wurde\autocite[][]{dnba}.\\
\linebreak
Da es sich bei dieser Arbeit in den technischen Abschnitten lediglich um die Dokumentation des Backends, also des Server-seitigen Codes handelt, ist es wichtig ebenfalls zu verstehen, wie die Studierenden der Hochschule Hof, also die primäre Nutzergruppe, von den Ergebnissen dieser Arbeit profitieren können. Das geschieht im Prototyp der Hochschul-\ac{App} durch eine web-basierte Nutzeroberfläche, die mit Hilfe des Typescript-Frameworks \textit{Angular} angefertigt wurde. Auch hierzu wurden umfangreiche Analysen zur Erstellung dieser und zur Kommunikation mit dem Backend erstellt, welche besonders auf die Sicherheitskonzepte dieses Prototypen eingehen. Die Analyse dieser Arbeit wurde in der Bachelorarbeit \textit{Web-basierte Hochschul-\ac{App} - Authentifizierung und Personalisierung} betrieben, die Dokumentation zur Implementierung in der gleichnamigen Praxisarbeit\autocites[][]{andreasba}[][]{andreaspa}.