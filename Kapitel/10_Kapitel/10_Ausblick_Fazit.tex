% Ausblick Fazit
\chapter{Ausblick und Fazit}
\label{sec:ausblick_fazit}

Zum Abschluss dieser Praxisarbeit zur Implementierung des Prototypen der neuen, web-basierten Hochschul-\ac{App} soll auf die großen Ziele dieser Neuimplementierung eingegangen werden. Hierbei werden nochmals die Beweggründe aufgelistet, die zur erneuten Entwicklung einer Anwendung geführt haben, die bereits in anderer Form bestand. Danach soll evaluiert werden, ob die gesetzten Ziele erreicht werden konnten.

\section{Ausblick\label{sec:ausblick}}

Wie in der parallel zu dieser Arbeit entstandenen Bachelorarbeit mehrfach geäußert wurde, existieren im Hochschulumfeld bereits mehrere Implementierungen einer allgemeinen Hochschul-\ac{App}. Die erneute Implementierung ist aus dem Wunsch heraus entstanden, eine plattformübergreifende Anwendung zu erschaffen, die die großen Probleme der bereits bestehenden Anwendungen eliminiert. Diese Probleme liegen besonders in der Plattformabhängigkeit der Anwendungen. Nachdem jede dieser Plattformen, darunter iOS, Android und Windows, spezielle Kenntnisse für die Entwicklung erfordern, die im allgemeinen Studium der Fakultät Informatik nicht zwingend gelehrt werden, wurde es immer schwerer, Entwickler zu finden, die das nötige Know-How hatten, diese \acp{App} am Leben zu erhalten. Mit der neuen Hochschul Anwendung wird dies in Zukunft kein Problem mehr sein. Das Backend, um das es in dieser Praxisarbeit primär geht, wurde in der Programmiersprache Java geschrieben, eine Sprache, die in allen Studiengängen der Fakultät Informatik ausgiebig gelehrt wird und deren aller Studierenden dieser Studiengänge mächtig sein sollten. Das dabei verwendete \textit{Spring Boot} Framework erleichtert das Arbeiten mit den komplexen Konzepten des Java \ac{EE} Umfelds ungemein. Um keine Probleme bei der Entwicklung der Oberfläche zu erzwingen, wurde eine \ac{REST}-Schnittstelle entwickelt, die Abhängigkeiten zur verwendeten Oberfläche komplett eliminiert. Dies garantiert die Zukunftsfähigkeit der Datenquelle und die dauerhafte Austauschbarkeit der Oberfläche.\\
\linebreak
Ein weiteres Problem der Nutzer der bestehenden Anwendungen ist der Mangel an neuen Funktionen. Durch fehlende Entwickler können Fehler und neue Funktionen an den plattformabhängigen Anwendungen nur schwer umgesetzt werden. Auch dieses Problem soll mit der neuen Hochschul Anwendung in Zukunft eliminiert sein. Einerseits bringt die Datenschnittstelle des \ac{REST} Servers bereits durch sein großes Potential an Kombinationen der Daten weit mehr Funktionen mit, als die alten Anwendungen der Hochschule, andererseits ermöglicht die Microservice Architektur der \ac{App} eine problemlose und extrem einfache Integration neuer Features. Diese Arbeit erläutert zudem, wie neue Funktionen von jedermann hinzugefügt werden können. Der Fokus auf die ständige Weiterentwicklung in Kombination mit der Robustheit der Anwendung steht klar im Zentrum dieses Projektes und wird in Zukunft auch den Erfolg der Anwendung garantieren.\\
\linebreak
Zu all den Vorteilen und Ausblicken, die bereits vorgestellt wurden, soll dennoch der Ursprung der Anwendung nicht vergessen werden. Sie ist im Rahmen einer Abschlussarbeit entstanden und lieferte somit eine Basis, auf der die Verfasser dieser Arbeit die Möglichkeit hatten, sich neues Wissen in den aktuellen Techniken im Bereich der web-basierten Programmierung anzueignen. Von dieser Möglichkeit sollen nun in Zukunft auch andere Studierende der Hochschule Hof und anderer Einrichtungen profitieren. Dies wird durch die Code Contributions und der Mitarbeitsmöglichkeiten ermöglicht. Zudem bietet die \ac{REST}-Schnittstelle ebenfalls eine Datenquelle, an denen Studierende testen und verstehen können, wie die moderne Kommunikation zwischen Clients und Servern abläuft.

\section{Fazit\label{sec:fazit}}

In dieser Praxisarbeit zum Thema der web-basierten Hochschul-\ac{App} wurde ein solides Fundament für eine erfolgreiche Produktionsumgebung der neuen Anwendung geschaffen. Entwicklerteams, die die Arbeit an der \ac{App} übernehmen, haben so die Möglichkeit, alles, das es über diese Software zu wissen gibt, nachzuverfolgen und zu verstehen. Im Rahmen der Rechtfertigung der verwendeten Ressourcen wie der Programmiersprache und des Microservice Frameworks wurden bereits wertvolle Erkenntnisse und Erklärungen dargestellt, welche den späteren Entwicklern die Einarbeitung in die Thematik deutlich vereinfachen werden. Des weiteren wurden die grundlegenden Konzepte und Konfigurationen der Anwendung und speziell des \textit{Spring Boot} Frameworks erläutert. Das ist besonders wichtig, um zu verstehen, welche Konfigurationen eventuelle Fehler verursachen könnten. Ebenso wichtig sind diese Informationen jedoch auch bei der Erstellung neuer Microservices, die den Funktionsumfang der Anwendung erweitern sollen, denn hierbei kann nach dem Vorbild der bereits vorhandenen Services vorgegangen werden.\\
\linebreak
Um dann produktiv an der Weiterentwicklung der neuen Hochschul-\ac{App} arbeiten zu können wurden den späteren Entwicklern einige hilfreiche und äußert wertvolle Werkzeuge an die Hand gegeben, die die Entwicklung an der Software deutlich vereinfachen. Hierzu wurde die \ac{IDE} \textit{IntelliJ IDEA} vorgestellt. Außerdem wurden die benötigten Einstellungen erklärt, welche zum Nutzen des vollen Funktionsumfangs dieser \ac{IDE} notwendig sind. Darauf aufbauend wurden dann die verschiedenen Umgebungen erläutert, in der die Anwendung laufen wird. Hierbei wurde besonders Wert darauf gelegt, die Unterschiede zwischen der Entwicklungs- und der Produktionsumgebung aufzuzeigen.\\
\linebreak
Abschließend wurde dann die eigentliche Anwendung genauer untersucht. Es wurde die Dokumentation zu den \ac{REST}-Endpunkten erklärt und gezeigt, wo diese zu finden ist und wie sie den Entwicklern bei der Arbeit hilft. Um die qualitativen Anforderungen der \ac{App} zu verdeutlichen wurde ebenfalls detailliert dargestellt, wie das Backend der Anwendung getestet wurde und im späteren Verlauf auch weiterhin getestet werden soll. Die Probleme, die beim Testen und bei der Entwicklung aufgetreten sind, wurden daraufhin ebenfalls dargestellt. Besonders für die spätere Suche nach Fehlern soll dieses Kapitel eine wertvolle Ressource darstellen. Jedoch sollen spätere Probleme nicht nur von den Entwicklern selbst behoben werden können. Nach dem Erläutern des Weiterentwicklungsplans wurde auch nochmals betont, welche Rolle jeder einzelne Nutzer in der Pflege und der Qualitätssicherung der Anwendung hat.\\
\linebreak
Allgemein kann man somit sagen, dass das große Ziel, eine neue Hochschul-\ac{App} zu schaffen, welche die Bedürfnisse eines jeden Studierenden der Hochschule Hof bestmöglich erfüllt, erreicht wurde. Mit dieser Arbeit soll der Prototyp nun abgeschlossen werden, womit der erste Kontakt der neuen Anwendung mit den Studierenden bevorsteht.
