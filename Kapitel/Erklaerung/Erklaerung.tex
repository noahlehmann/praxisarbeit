\chapter*{Eidesstattliche Erklärung}
\addcontentsline{toc}{chapter}{Eidesstattliche Erklärung}
Ich erkläre hiermit, dass ich meinen Beitrag zur vorliegenden Gruppenarbeit (Kapitel 1, 2.2, 3.1, 3.3, 3.4, 4.1, 4.5, 4.6, 5.2, 6.1, 6.2, 6.3, 7.4, 8.1, 9.1, 9.2, 10) selbständig und ohne Benutzung anderer als der angegebenen Hilfsmittel angefertigt habe; das gleiche gilt für die von den auf dem Titelblatt der Arbeit genannten Autoren gemeinsam verfassten Teile (Kapitel 2.1, 3.2, 3.5, 3.6, 3.7, 4.2, 4.3, 4.5, 4.7, 5.1, 6.4, 6.5, 7.1, 7.2, 7.3, 8.2, 8.3, 9.3, 9.4, 9.5). 
Die aus fremden Quellen direkt oder indirekt übernommenen Gedanken sind als solche kenntlich gemacht.\\
\linebreak
Die Arbeit wurde nach meiner besten Kenntnis bisher in gleicher oder ähnlicher Form keiner anderen Prüfungsbehörde vorgelegt und auch noch nicht veröffentlicht.\\
\linebreak
\linebreak
\linebreak
Hof, den 29.01.2020


%(Kapitel 2.1, 3.2, 3.5, 3.6, 3.7, 4.2, 4.3, 4.5, 4.7, 5.1, 6.4, 6.5, 7.1, 7.2, 7.3, 8.2, 8.3, 9.3, 9.4, 9.5) %Dennis
%(Kapitel 1, 2.2, 3.1, 3.3, 3.4, 4.1, 4.5, 4.6, 5.2, 6.1, 6.2, 6.3, 7.4, 8.1, 9.1, 9.2, 10) %Noah

%\newpage

%\chapter*{Abstract}
%
%In der Praxisarbeit zum Projekt des Prototypen der neuen web-basierten Hoch\-schul-\ac{App} werden wertvolle Erkenntnisse und Erklärungen dargestellt, welche den späteren Entwicklern die Einarbeitung das besagte Projekt deutlich vereinfachen sollen. Es werden die grundlegenden Konzepte und Konfigurationen der Anwendung und speziell des \textit{Spring Boot} Frameworks erläutert, welches für die Implementierung des Backends ausgewählt wurde. Hierzu werden ebenfalls Analysen zur Wahl des Frameworks und der Programmiersprache angefertigt.\\
%\linebreak
%Um produktiv an der Weiterentwicklung der neuen Hochschul-\ac{App} arbeiten zu können werden den späteren Entwicklern einige hilfreiche und äußert wertvolle Werkzeuge an die Hand gegeben, die die Entwicklung an der Software deutlich vereinfachen. Hierzu wird die \ac{IDE} \textit{IntelliJ IDEA} vorgestellt. Außerdem werden die benötigten Einstellungen erklärt, welche zum Nutzen des vollen Funktionsumfangs dieser \ac{IDE} notwendig sind. Darauf aufbauend werden dann die verschiedenen Umgebungen erläutert, in der die Anwendung laufen wird. Hierbei wird besonders Wert darauf gelegt, die Unterschiede zwischen der Entwicklungs- und der Produktionsumgebung aufzuzeigen.\\
%\linebreak
%Abschließend wird die eigentliche Anwendung genauer untersucht. Es wird die Dokumentation zu den \ac{REST}-Endpunkten erklärt und gezeigt, wo diese zu finden ist und wie sie den Entwicklern bei der Arbeit hilft. Um die qualitativen Anforderungen der \ac{App} zu verdeutlichen wird ebenfalls detailliert dargestellt, wie das Backend der Anwendung getestet wurde und im späteren Verlauf auch weiterhin getestet werden soll. Die Probleme, die beim Testen und bei der Entwicklung aufgetreten sind, werden daraufhin ebenfalls dargestellt. Besonders für die spätere Suche nach Fehlern soll dieses Kapitel eine wertvolle Ressource darstellen. Nach dem Erläutern des Weiterentwicklungsplans wird auch nochmals betont, welche Rolle jeder einzelne Nutzer in der Pflege und der Qualitätssicherung der Anwendung hat.
%